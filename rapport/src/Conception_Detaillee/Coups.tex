%Conecption detaillée
\subsection{Coups}
	\subsubsection{Type}
	
	Coups = ListeChainee\textless Coup\textgreater . \\
	Le TAD ListeChainee et ses fonctions sont ceux présents dans le cours.

	\subsubsection{Fonction TAD}
		\paragraph{creerCoups}
			\begin{algorithme}
				\small
				\fonction
				{creerCoups}
				{}
				{Coups}
				{}
				{
					{\retourner{liste()}}
				}
			\end{algorithme}

	\paragraph{obtenirCoup}
		\begin{algorithme}
			\small
			\fonctionAvecPreconditions
			{obtenirCoup}
			{coups: Coups, position : \naturel}
			{Coup}
			{position \textgreater{} 0 et position $\leq$ longueur(coups)}
			{
				temp : Coups \\
				i : \naturel
			}
			{
				\affecter{i}{1}
				\tantque{i<position et non estVide(temp)}
				{
					\affecter{temp}{obtenirListeSuivante(coups)}
				}
				\retourner{obtenirElement(temp)}
			}
		\end{algorithme}

	\paragraph{ajouterCoup}
		\begin{algorithme}
			\small
			\procedure
			{ajouterCoup}
			{\paramEntreeSortie{coups : Liste<coup>} \paramEntree{coup : coup} }
			{}
			{
				\instruction{ajouter(coups, coup)}
			}

		\end{algorithme}
		
		\paragraph{estVide}
			\begin{algorithme}
				\small
				\fonction
				{Coups\textunderscore estVide}
				{coups : Coups}
				{\booleen}
				{}
				{
					\retourner{estVide(coups)}
				}
			\end{algorithme}