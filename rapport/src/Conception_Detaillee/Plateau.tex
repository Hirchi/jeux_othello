\subsection{Plateau}
\subsubsection{Type}
\begin{algorithme}
			\begin{enregistrement}{Plateau}
				\champEnregistrement{largeur}{\naturel}
				\champEnregistrement{hauteur}{\naturel}
\champEnregistrement{tableauCellule}{\tableauDeuxDimensions{1..MAX}{1..MAX}{de}{Cellule}}
			\end{enregistrement}
		\end{algorithme}

 
\subsubsection{Fonctions TAD}
\paragraph{creerPlateau}
\begin{algorithme}
\fonction
{creerPlateau}
{largeur, hauteur : \naturel}
{Plateau}
{
i, j : \naturel \newline
posTmp : Position \newline
celluleTmp : Cellule \newline
tableauTmp : \tableauDeuxDimensions{1..MAX}{1..MAX}{de}{Cellule} \newline
p : Plateau
}
{
\affecter{p.largeur}{largeur}
\affecter{p.hauteur}{hauteur}
\pour{i}{1}{hauteur}{}
	{
		\pour{j}{1}{largeur}{}
		{
			\affecter{posTmp}{creerPosition(i, j)}
			\affecter{celluleTmp}{creerCellule(i, j)}
			\affecter{p.tableauCellule}{cellule(i, j)}
		}
	}
\retourner{p}
}
\end{algorithme}



\paragraph{initialiserPionDebut}
\begin{algorithme}
				\small
				\procedure
				{assignerNom}
				{\paramEntreeSortie{joueur : Joueur}}
				{}
				{
\instruction{placerCouleur(obtenirCellule(p, position(4, 4)), blanc())}
	\instruction{placerCouleur(obtenirCellule(p, position(5, 4)), noir())}
	\instruction{placerCouleur(obtenirCellule(p, position(5, 5)), blanc())}
	\instruction{placerCouleur(obtenirCellule(p, position(4, 5)), noir())}
				}
\end{algorithme}


\paragraph{obtenirCellules}
\begin{algorithme}
\fonction
{obtenirCellules}
{p : Plateau}
{\tableauUneDimension{64}{de}{Cellule}}
{
i, j : \naturel \newline
res : \tableauUneDimension{64}{de}{Cellule}
}
{
\pour{i}{1}{obtenirHauteur(p)}{}
	{
		\pour{j}{1}{obtenirLargeur(p)}{}
		{
			\affecter{res[i]}{obtenirCellule(p,creerPosition(i,j)}
		}
	}

	\retourner{res}
}
\end{algorithme}


\paragraph{obtenirCellule}
\begin{algorithme}
\fonction
{obtenirCellule}
{p : Plateau \newline
pos : Position}
{Cellule}
{
i, j : \naturel
}
{
	\retourner{p.tableauCellule(abscisse(pos),
	ordonnee(pos))}
}
\end{algorithme}


\paragraph{estRempli}
\begin{algorithme}
\fonction
{estRempli}
{p : Plateau}
{\booleen}
{
i, j : \naturel \newline
estRempli  : \booleen
}
{
	\affecter{estRempli}{Vrai}
	\affecter{i}{1}
	\affecter{j}{1}
	\tantque{estRempli et (i $<=$ obtenirHauteur(p) ou j $<=$ obtenirLargeur(p))}
		{
		\affecter{celluleTmp}{obtenirCellule(plateau,creerPosition(i,j))}
		\sialors{estVide(celluleTmp)}
			{
			\affecter{estRempli}{Faux}
			}
		\sialorssinon{j $<=$ obtenirLargeur(p)}
			{
			 \affecter{j}{j+1}
			}
			{
			 \affecter{i}{i+1}
			}
		}
		
	\retourner{estRempli}
}
\end{algorithme}


\paragraph{jouerCoup}
\begin{algorithme}
\procedure
{jouerCoup}
{\paramEntreeSortie{p : Plateau} \paramEntree{c : Coup}}
{x,y,i : \naturel \newline
cellulesRetournees : Cellule \newline
celluleActuelle : Cellule}
{
	\affecter{x}{abscisse(obtenirPosition(c))}
	\affecter{y}{ordonnee(obtenirPosition(c))}
	\instruction{placerPion(obtenirCellule(position(x,y)), obtenirPion(c))}
	\affecter{cellulesRetournees}{obtenirCellulesRetournees(p, position(x,y))}
	\pour{i}{1}{longueur(cellulesRetournees)}{}
	{
		\affecter{celluleActuelle}{obtenirCellule(cellulesRetournees, i)}
		\instruction{placerCouleur(celluleActuelle, complement(obtenirCouleur(celluleActuelle))}
	}
} 

\end{algorithme}



\paragraph{estUnCoupPossible}
\begin{algorithme}
\fonction
{estUnCoupPossible}
{p : Plateau, c : Coup}
{\booleen}
{pos, posTmp : Position \newline
 couleur,complement : Couleur \newline
 directions : liste$<Direction>$ \newline
 res : \booleen \newline
 i : \naturel
}
{
	\affecter{pos}{obtenirPosition(c)}
	\affecter{couleur}{obtenirCouleur(c)}
	\affecter{complement}{complement(couleur)}
	\sialors{obtenirCouleur(p,pos) $<>$ sansCouleur()}
	{
		\retourner{Faux}
	}
	\affecter{res}{Faux}
	\affecter{i}{0}
	\affecter{directions}{${NORD,NORDEST,EST,SUDEST,SUD,SUDOUEST,
	OUEST,NORDOUEST}$}
	\tantque{res et i $<$ 8}
	{
		\affecter{posTmp}{positionAdjacente(pos,directions[i]}
		\sialors{estDanslePlateau(p,posTmp)}
		{
			\sialors{obtenirCouleur(p,posTmp) = complement}
			{
				\affecter{res}{estUnCoupPossibleR(p,posTmp,Direction[i],couleur)}
			}
		}
	}
	\retourner{res}
}
\end{algorithme}

\paragraph{estUnCoupPossibleR}
\begin{algorithme}
\fonction
{estUnCoupPossibleR}
{p : Plateau, direc : Direction,pos : Position, coulJoueur : Couleur}
{\booleen}
{posAdj : Position \newline
 couleurAdj : Couleur \newline
}
{
	\affecter{posAdj}{positionAdjacente(pos,direct)}
	\sialorssinon{estDanslePlateau(p,posAdj)}
	{
		\affecter{couleurAdj}{obtenirCouleur(p,posAdj)}
		\sialorssinon{couleurAdj = coulJoueur}
		{
			\retourner{Vrai}
		}
		{
			\sialorssinon{estVide
			(obtenirCellule(p,posAdj))}
			{
				\retourner{Faux}
			}
			{
				\retourner{estUnCoupPossibleR(p,posAdj,direct,coulJoueur)}
			}
		}
	}
	{
		\retourner{Faux}
	}

}
\end{algorithme}





\paragraph{nombrePions}
\begin{algorithme}
\fonction
{nombrePions}
{p : Plateau, joueurRef : Joueur}
{\naturel}
{i,j, nbPions : \naturel \newline
 celluleActuelle : Cellule}
{
	\affecter{nbPions}{0}
	\pour{i}{1}{p.hauteur}{}
	{
		\pour{j}{1}{p.largeur}{}
		{
			\affecter{celluleActuelle}{obtenirCellule(p, position(i,j)}
			\sialors{non estVide(celluleActuelle) et obtenirCouleur(celluleActuelle) = obtenirCouleur(joueurRef)}
			{
				\affecter{nbPions}{nbPions + 1}
			} 
		}
	}
	\retourner{nbPions}
}
\end{algorithme}

\paragraph{obtenirLargeur}
\begin{algorithme}
\fonction
{obtenirLargeur}
{p : Plateau}
{\naturel}
{
}
{
	\retourner{p.largeur}
}
\end{algorithme}

\paragraph{obtenirHauteur}
\begin{algorithme}
\fonction
{obtenirHauteur}
{p : Plateau}
{\naturel}
{
}
{
	\retourner{p.hauteur}
}
\end{algorithme}


\paragraph{detruirePlateau}
\begin{algorithme}
\procedure
{detruirePlateau}
{\paramEntreeSortie{p : Plateau}}
{}
{
	\affecter{p}{creerPlateau(0,0)}
} 

\end{algorithme}


\paragraph{obtenirCouleur}
\begin{algorithme}
\fonction
{obtenirCouleur}
{p : Plateau; pos : Position}
{Couleur}
{
}
{
	\retourner{obtenirCouleur(obtenirPosition(p,pos))}
}
\end{algorithme}

\paragraph{estDanslePlateau}
\begin{algorithme}
\fonction
{estDanslePlateau}
{p : Plateau; pos : Position}
{\naturel}
{
}
{
	\affecter{abscisse}{abscisse(pos)}
	\affecter{ordonnee}{ordonnee(pos)}
	\affecter{hauteur}{obtenirHauteur(p)}
	\affecter{largeur}{obtenirLargeur(p)}
	\retourner{1 $<=$ abscisse $<=$ largeur et 1 $<=$ ordonnee $<=$ hauteur }
}
\end{algorithme}

\paragraph{listeCoupPossible}
\begin{algorithme}
\fonction
{listeCoupPossible}
{p : Plateau; couleurRef : Couleur}
{Coups}
{res : Coups \newline
 i , j : \naturel \newline
 coupTmp : Coup
}
{
	\affecter{res}{CoupCreerVide()}
	\pour{i}{1}{obtenirHauteur(p)}{}
	{
		\pour{j}{1}{obtenirLargeur(p)}{}
		{
			\affecter{coupTmp}{creerCoup(creerPosition(i,j),couleurRef)}
			\sialors{estUnCoupPossible(p,coupTmp)}
			{
				\instruction{ajouterCoup(res,coupTmp)}
			} 
		}
	}
	
	\retourner{res}
}
\end{algorithme}

\paragraph{plateauEnString}
~\\
On suppose disposer d'une fonction naturelEnChaine(n) qui retourne la représentation d'un naturel en chaine de caracteres. \\

\begin{algorithme}
\fonction
{plateauEnString}
{plateau : Plateau}
{\chaine}
{
	hauteur, largeur, i, j: \naturel \\
	couleur : Couleur \\
	chaine : \chaine
}
{
	\affecter{hauteur}{obtenirHauteur(plateau)}
	\affecter{nbChiffresHauteur}{nbChiffres(hauteur)}
	\affecter{largeur}{obtenirLargeur(plateau)}
	\affecter{nbChiffresLargeur}{nbChiffres(largeur)}
	\affecter{chaine}{""}
	
	\instruction{ecrire(chaine, naturelEnChaine(hauteur))}
	\instruction{ecrire(chaine, ':')}
	\instruction{ecrire(chaine, naturelEnChaine(largeur))}
	
	\pour{i}{1}{hauteur}{}
	{
		\pour{j}{1}{largeur}{}
		{
			\affecter{couleur}{obtenirCouleur(obtenirCellule(plateau, i, j)}
			\instruction{ecrire(chaine, couleurVersChar(couleur)}
		}	
	}
	\retourner{chaine}

} 
\end{algorithme}

\paragraph{stringEnPlateau} ~\\
On suppose disposer d'une fonction niemeCaractere(chaine, n) qui retourne le n-ieme caractere de la chaine indiquée. \\

On suppose également disposer d'une fonction sousChaine(chaine, position) qui retourne le contenu de la chaîne indiquée à partir de la position indiquée jusqu'à la fin de la chaîne. \\

\begin{algorithme}
\fonction
{stringEnPlateau}
{string : \chaine}
{Plateau}
{
	hauteur, largeur, nbChiffresHauteur, nbChiffresLargeur, i, j, finDimensions: \naturel \\
	caractereActuel : Caractere \\
	plateau : Plateau
}
{
	\affecter{hauteur}{prochainNombreDansChaine(string)}
	\affecter{nbChiffresHauteur}{nbChiffres(hauteur)}
	\affecter{largeur}{prochainNombreDansChaine(sousChaine(string, nombreChiffresHauteur+1))}
	\affecter{nbChiffresLargeur}{nbChiffres(largeur)}
	
	\affecter{finDimensions}{nbChiffresLargeur+nbChiffresHauteur+2}
	\affecter{plateau}{creerPlateau(largeur, hauteur)}
	\pour{i}{1}{hauteur}{}
	{
		\pour{j}{1}{largeur}{}
		{
			\affecter{caractereActuel}{niemeCaractere(chaine, finDimensions+(j-1)+(i-1)*largeur)}
			\sialors{caractereActuel $\neq$ '.'}
			{
				\instruction{placerCouleur(obtenirCellule(plateau, creerPosition(j, i)), charVersCouleur(caractereActuel))} 
			}
		}
	}
	\retourner{plateau}
	

} 
\end{algorithme}

\subsubsection{Fonctions privées}

\paragraph{nbChiffres}
\begin{algorithme}
\fonction
{nbChiffres}
{n : \naturel}
{\naturel}
{nombreChiffres : \naturel}
{
	\affecter{nombreChiffres}{0}
	\tantque{n > 0}
	{
		\affecter{n}{n div 10}
		\affecter{nombreChiffres}{nombreChiffres + 1}		
	}
	\retourner{nombreChiffres}
}
\end{algorithme}

\paragraph{prochainNombreDansChaine}
~\\
On suppose que les chiffres de 0 à 9 se suivent dans le systeme de codage des caracteres utilisés (comme c'est le cas en ASCII).\\
\begin{algorithme}
\fonction
{prochainNombreDansChaine}
{string : \chaine}
{\naturel}
{
	i, nombre : \naturel \\
	debut : \booleen
}
{
	\affecter{debut}{Vrai}
	\affecter{nombre}{0}
	\affecter{i}{1}
	\tantque{niemeCaractere(string, i) $\neq$ FINCHAINE et (estUnChiffre(niemeCaractere(string, i)) ou debut)}
	{
		\sialors{estUnChiffre(niemeCaractere(string, i))}
		{
			\affecter{debut}{Faux}
			\affecter{nombre}{nombre*10}
			\affecter{nombre}{nombre+caractereEnNaturel(niemeCaractere) - ord('0')}
		}
		\affecter{i}{i+1}
	}
	\retourner{nombre}
}
\end{algorithme}

\paragraph{estUnChiffre}
\begin{algorithme}
\fonction
{estUnChiffre}
{c : Caractere}
{\booleen}
{}
{\retourner{ord(c) $\geq$ ord('0') et ord(c) $\leq$ ord('9')}}
\end{algorithme}
