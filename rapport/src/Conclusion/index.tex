\pagebreak
\section{Conclusion}

\subsection{Conclusion}
Ce projet nous a permis d'améliorer nos compétences en algorithmique et codage ainsi que nos compétences à tenir des délais et à travailler en groupe.\\

Nous sommes fiers d'avoir réussi à produire un Othello qui fonctionne et qui respecte les exigences imposées. De plus la base de ce dernier est évolutive, on pourrait ainsi imaginer un Othello sur un plus grand plateau ou avec plus de joueurs, et comme les TAD ont été conçus pour pouvoir supporter plus que deux joueurs ou un plateau de 8*8, il serait aisé d'implémenter de telles variantes en un minimum de modifications. \\

Cependant, si nous avions eu plus de temps, nous aurions aimé améliorer l'interface. Réaliser une interface qui ne serait pas sur le terminal et afficher le score au cours de la partie. On aurait aussi aimer coder un mode joueur contre joueur afin de pouvoir jouer les uns contre les autres. Finalement, on aurait pu améliorer encore plus notre IA afin qu'elle soit plus performante.

\subsection{Répartition des tâches}

\noindent HA : Hirchi ABDELHADI\\
ML : Marie LANTERNIER\\
GLG : Gawein LE GOFF\\
LP : Léo PACARY\\

\begin{tabular}{|c|c|c|c|c|}
\hline 
  & HA & ML & GLG & LP \\ 
\hline 
Analyse Descendante / Specification TAD & 1h & 7h & 3h & 8h \\ 
\hline 
Conception Préliminaire & 1h & 2h & 1h30 & 4h \\ 
\hline 
Conception Détaillée & 1h & 5h & 5h & 5h \\ 
\hline 
Implémentation & 0h & 5h & 30h & 40h \\ 
\hline 
Tests Unitaires & 1h & 20h & 5h & 5h \\ 
\hline 
\end{tabular} 


